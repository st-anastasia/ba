%%%%%%%%%%%%%%%%%%%%%%%%%%%%%%%%%%%%%%%%%%%%%%%%%
%---- LaTeX-Header fuer Abschlussarbeiten, Prof. Thomas Goerne, Dez. 2012/Aug. 2013 ----
%%%%%%%%%%%%%%%%%%%%%%%%%%%%%%%%%%%%%%%%%%%%%%%%%

\documentclass[12pt,paper=A4,pointlessnumbers,bibtotoc,liststotoc,DIV=11,BCOR=1mm,halfparskip]{scrreprt}
% BCOR ist die Bindekorrektur (verlorener Rand am linken Blattrand)! Wert haengt von der Art der Heftung ab!!
% DIV ist eine Satzspiegeleinstellung von KOMA-Script / sccreprt.
\hyphenpenalty = 1000
\usepackage[a4paper,
left=2.5cm, right=2.5cm,
top=2.5cm, bottom=2.5cm]{geometry}

\pagestyle{headings}
\usepackage[utf8]{inputenc}
\usepackage[T1]{fontenc} % Font Encoding fuer europaeische Schriften mit Umlauten (Unterstuetzung der Worttrennung)
\usepackage{lmodern} % PostScript-Varianten der TeX Computer Modern-Schriften laden
\usepackage[english,ngerman]{babel} % Spracheinstellungen fuer Englisch und Neudeutsch laden
\usepackage[autostyle=true,german=quotes]{csquotes}
\usepackage{times}                                     % Schriften Paket
\usepackage{array,ragged2e}                            % Wichtig für Abstandsformatierung
\usepackage{cmbright}                                  % serifenlose Schrift als Standard + alle für TeX
                                                       % benötigten mathematischen Schriften einschließlich der AMS-Symbole
\usepackage[scaled=.90]{helvet}                        % skalierte Helvetica als \sfdefault
\usepackage{courier}                                   % Courier als \ttdefault
\usepackage[automark]{scrpage2}                        % Anpassung der Kopf- und Fußzeilen
\usepackage{xspace}                                    % Korrekter Leerraum nach Befehlsdefinitionen
\usepackage[onehalfspacing]{setspace}                                % 1.5 zeilen abstand

\usepackage{graphicx} % Grafikeinbindung (fuer .JPG, .JPEG, .PNG und .PDF, falls pdflatex benutzt wird)
\usepackage[table]{xcolor} % ermoeglicht farbige Schrift und farbige Tabellenzeilen
\definecolor{black}{gray}{0} % Umdefinition der Farbe black, falls noetig (0=schwarz, 1=weiss)
\definecolor{dblue}{rgb}{0.1,0.2,0.6} % Dunkelblau, fuer Hyperlinks
\definecolor{lgray}{gray}{0.9} % Hellgrau, fuer Tabellen (0=schwarz, 1=weiss)

\usepackage[final]{pdfpages}                           % include pages of external PDF documents
\usepackage{tabularx}                                  % Spaltenbreite bis zur Seitenbreite dehnen

\usepackage{booktabs} % fuer schoene Tabellen

\usepackage[round,authoryear]{natbib} % Literaturverweise mit Name/Jahreszahl in runden Klammern
\bibpunct[:\,]{(}{)}{,}{a}{}{,~}  % Feinformatierung der Natbib-Zitierweise

\usepackage[hyphens]{url}
\usepackage[colorlinks=true,linkcolor=dblue,citecolor=dblue,urlcolor=dblue]{hyperref}
% die Pakete url und hyperref ermoeglichen anklickbare URLs im Quellenverzeichnis in definierter Farbe,
% sie ermoeglichen den Zeilenumbruch bei langen URLs, und sie erzeugen Hyperlinks (Farbe s.o.)
% zwischen Quellenverweis und Quellenverzeichnis sowie zwischen label und ref im PDF-Dokument

% Fonteinstellungen fuer Bildunterschriften: Unterschrift serifenlos, "Abbildung" fett (bfseries = bold face series)
\setkomafont{captionlabel}{\sffamily\bfseries}
\setkomafont{caption}{\sffamily}

\usepackage{makeidx}
\usepackage[cache=false]{minted}
\setminted{fontsize=\small,baselinestretch=1}
\renewcommand{\listingscaption}{Quellcode}
\renewcommand{\listoflistingscaption}{Quellcode Verzeichnis}
\usepackage{multirow}
\usepackage[xindy]{glossaries}
\newglossaryentry{databinding}
{
  name = Databinding,
  description = {  Eine Teschnik, welche die Benutzeroberfläche der App mit den dort angezeigten Daten verbindet und synchronisiert }
}

\newglossaryentry{json}
{
  name = JSON,
  description = { Javascript Object Notation. \url{http://json.org/} }
}

\newglossaryentry{di}
{
  name = Dependency Injection,
  description = { Ein Entwurfsmuster der objektorientierten Programmierung, bei dem Abhängigkeit zwischen Objekten erst zur Laufzeit hergestellt werden }
}

\newglossaryentry{ng-directive}
{
  name = AngularJS Direktive,
  description = { Marker auf dem DOM Element, welcher dem AngularJS Renderer mitteilt spezifisches Verhalten an das DOM Element zu binden. \url{https://docs.angularjs.org/guide/directive}}
}

\newglossaryentry{ng-service}
{
  name = Angular Service,
  description = { Ersetzbare Objekte, welche zusammengebunden werden um Code zu organisieren und zwischen Modulen zu teilen. \url{https://docs.angularjs.org/guide/services}}
}

\newglossaryentry{ng-factory}
{
  name = Angular Factory,
  description = { Komponenten, welche Services erzeugen. \url{https://docs.angularjs.org/guide/providers\#factory-recipe}}
}


\newglossaryentry{pug}
{
  name = Pug,
  description = { Javascript HTML Template Engine. \url{https://pugjs.org/} }
}

\newglossaryentry{transpiler}
{
  name = Transpiler,
  description = {Source to Source Compiler}
}

\newglossaryentry{localstorage}
{
  name = LocalStorage,
  description = { HTML5 Web Storage Schnittstelle. \url{https://www.w3schools.com/html/html5\_webstorage.asp} }
}


\newglossaryentry{webpack}
{
  name = Webpack,
  description = { Javascript module bundler. \url{https://webpack.js.org/}}
}

\newglossaryentry{es2016}
{
  name = ES2016,
  description = { Spezifische Version des ECMAScript standardisierten Sprachkerns von JavaScript. \url{http://www.ecma-international.org/ecma-262/7.0/}}
}

\newglossaryentry{sass}{
  name = SASS,
  description = { Stylesheet-Sprache, die als CSS-Präprozessor, mit Variablen, Schleifen und vielen anderen Funktionen, die Cascading Style Sheets (CSS) nicht mitbringen, die Erstellung von CSS vereinfacht und die Pflege großer Stylesheets erleichtert }
}

\newglossaryentry{jwt}
{
  name = JWT,
  description = {Jason Web Token. JSON Web Tokens sind eine offene Industriestandard RFC 7519 Methode zur sicheren Representierung von Forderungen zwischen zwei Parteien }
}

\newglossaryentry{promise}
{
  name = Promise,
  description = {Eine Repräsentation einer eventuellen Ausführung einer asynchronen Operation}

}

\makeglossaries
\usepackage[xindy]{imakeidx}
\makeindex


%------------------------------------------------------------------------------------------------------------------
%------ Eigenstaendigkeitserklaerung im gerahmten Kasten (parbox in einer framebox) ------
%------------------------------------------------------------------------------------------------------------------

\newcommand{\eigen}{
\setlength{\fboxsep}{2ex}
\setlength{\fboxrule}{0.8pt}
% Einstellungen fuer Rahmenabstand und Rahmendicke der Framebox
\begin{center}
  \fbox{
    \parbox{0.8\linewidth}{
    Ich versichere, die vorliegende Arbeit selbstst\"andig ohne fremde Hilfe verfasst
    und keine anderen Quellen und Hilfsmittel als die angegebenen benutzt zu haben.
    Die aus anderen Werken w\"ortlich entnommenen Stellen oder dem Sinn nach
    entlehnten Passagen sind durch Quellenangaben eindeutig kenntlich gemacht.
    \par\bigskip\bigskip\bigskip\bigskip
    \hspace*{0.8cm}Ort, Datum \hfill \vorname~\nachname\hspace*{0.8cm}
    }
  }
\end{center}
}

%%%%%%%%%%%%%%%%%%%%%%%%%%%%%%%%%%%%%%%%%%%%%%%%%