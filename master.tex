%%%%%%%%%%%%%%%%%%%%%%%%%%%%%%%%%%%%%%%%%%%%%%%%%
%------ LaTeX-Template für Abschlussarbeiten, Prof. Thomas Görne, Dezember 2012 --------
%%%%%%%%%%%%%%%%%%%%%%%%%%%%%%%%%%%%%%%%%%%%%%%%%

%---- Header (mit Formateinstellugen) laden, Inputencoding prüfen ------

%%%%%%%%%%%%%%%%%%%%%%%%%%%%%%%%%%%%%%%%%%%%%%%%%
%---- LaTeX-Header fuer Abschlussarbeiten, Prof. Thomas Goerne, Dez. 2012/Aug. 2013 ----
%%%%%%%%%%%%%%%%%%%%%%%%%%%%%%%%%%%%%%%%%%%%%%%%%

\documentclass[12pt,paper=A4,pointlessnumbers,bibtotoc,liststotoc,DIV=11,BCOR=1mm,halfparskip]{scrreprt}
% BCOR ist die Bindekorrektur (verlorener Rand am linken Blattrand)! Wert haengt von der Art der Heftung ab!!
% DIV ist eine Satzspiegeleinstellung von KOMA-Script / sccreprt.
\hyphenpenalty = 1000
\usepackage[a4paper,
left=2.5cm, right=2.5cm,
top=2.5cm, bottom=2.5cm]{geometry}

\pagestyle{headings}
\usepackage[utf8]{inputenc}
\usepackage[T1]{fontenc} % Font Encoding fuer europaeische Schriften mit Umlauten (Unterstuetzung der Worttrennung)
\usepackage{lmodern} % PostScript-Varianten der TeX Computer Modern-Schriften laden
\usepackage[english,ngerman]{babel} % Spracheinstellungen fuer Englisch und Neudeutsch laden
\usepackage[autostyle=true,german=quotes]{csquotes}
\usepackage{times}                                     % Schriften Paket
\usepackage{array,ragged2e}                            % Wichtig für Abstandsformatierung
\usepackage{cmbright}                                  % serifenlose Schrift als Standard + alle für TeX
                                                       % benötigten mathematischen Schriften einschließlich der AMS-Symbole
\usepackage[scaled=.90]{helvet}                        % skalierte Helvetica als \sfdefault
\usepackage{courier}                                   % Courier als \ttdefault
\usepackage[automark]{scrpage2}                        % Anpassung der Kopf- und Fußzeilen
\usepackage{xspace}                                    % Korrekter Leerraum nach Befehlsdefinitionen
\usepackage[onehalfspacing]{setspace}                                % 1.5 zeilen abstand

\usepackage{graphicx} % Grafikeinbindung (fuer .JPG, .JPEG, .PNG und .PDF, falls pdflatex benutzt wird)
\usepackage[table]{xcolor} % ermoeglicht farbige Schrift und farbige Tabellenzeilen
\definecolor{black}{gray}{0} % Umdefinition der Farbe black, falls noetig (0=schwarz, 1=weiss)
\definecolor{dblue}{rgb}{0.1,0.2,0.6} % Dunkelblau, fuer Hyperlinks
\definecolor{lgray}{gray}{0.9} % Hellgrau, fuer Tabellen (0=schwarz, 1=weiss)

\usepackage[final]{pdfpages}                           % include pages of external PDF documents
\usepackage{tabularx}                                  % Spaltenbreite bis zur Seitenbreite dehnen

\usepackage{booktabs} % fuer schoene Tabellen

\usepackage[round,authoryear]{natbib} % Literaturverweise mit Name/Jahreszahl in runden Klammern
\bibpunct[:\,]{(}{)}{,}{a}{}{,~}  % Feinformatierung der Natbib-Zitierweise

\usepackage[hyphens]{url}
\usepackage[colorlinks=true,linkcolor=dblue,citecolor=dblue,urlcolor=dblue]{hyperref}
% die Pakete url und hyperref ermoeglichen anklickbare URLs im Quellenverzeichnis in definierter Farbe,
% sie ermoeglichen den Zeilenumbruch bei langen URLs, und sie erzeugen Hyperlinks (Farbe s.o.)
% zwischen Quellenverweis und Quellenverzeichnis sowie zwischen label und ref im PDF-Dokument

% Fonteinstellungen fuer Bildunterschriften: Unterschrift serifenlos, "Abbildung" fett (bfseries = bold face series)
\setkomafont{captionlabel}{\sffamily\bfseries}
\setkomafont{caption}{\sffamily}

\usepackage{makeidx}
\usepackage[cache=false]{minted}
\setminted{fontsize=\small,baselinestretch=1}
\renewcommand{\listingscaption}{Quellcode}
\renewcommand{\listoflistingscaption}{Quellcode Verzeichnis}
\usepackage{multirow}
\usepackage[xindy]{glossaries}
\newglossaryentry{databinding}
{
  name = Databinding,
  description = {  Eine Teschnik, welche die Benutzeroberfläche der App mit den dort angezeigten Daten verbindet und synchronisiert }
}

\newglossaryentry{json}
{
  name = JSON,
  description = { Javascript Object Notation. \url{http://json.org/} }
}

\newglossaryentry{di}
{
  name = Dependency Injection,
  description = { Ein Entwurfsmuster der objektorientierten Programmierung, bei dem Abhängigkeit zwischen Objekten erst zur Laufzeit hergestellt werden }
}

\newglossaryentry{ng-directive}
{
  name = AngularJS Direktive,
  description = { Marker auf dem DOM Element, welcher dem AngularJS Renderer mitteilt spezifisches Verhalten an das DOM Element zu binden. \url{https://docs.angularjs.org/guide/directive}}
}

\newglossaryentry{ng-service}
{
  name = Angular Service,
  description = { Ersetzbare Objekte, welche zusammengebunden werden um Code zu organisieren und zwischen Modulen zu teilen. \url{https://docs.angularjs.org/guide/services}}
}

\newglossaryentry{ng-factory}
{
  name = Angular Factory,
  description = { Komponenten, welche Services erzeugen. \url{https://docs.angularjs.org/guide/providers\#factory-recipe}}
}


\newglossaryentry{pug}
{
  name = Pug,
  description = { Javascript HTML Template Engine. \url{https://pugjs.org/} }
}

\newglossaryentry{transpiler}
{
  name = Transpiler,
  description = {Source to Source Compiler}
}

\newglossaryentry{localstorage}
{
  name = LocalStorage,
  description = { HTML5 Web Storage Schnittstelle. \url{https://www.w3schools.com/html/html5\_webstorage.asp} }
}


\newglossaryentry{webpack}
{
  name = Webpack,
  description = { Javascript module bundler. \url{https://webpack.js.org/}}
}

\newglossaryentry{es2016}
{
  name = ES2016,
  description = { Spezifische Version des ECMAScript standardisierten Sprachkerns von JavaScript. \url{http://www.ecma-international.org/ecma-262/7.0/}}
}

\newglossaryentry{sass}{
  name = SASS,
  description = { Stylesheet-Sprache, die als CSS-Präprozessor, mit Variablen, Schleifen und vielen anderen Funktionen, die Cascading Style Sheets (CSS) nicht mitbringen, die Erstellung von CSS vereinfacht und die Pflege großer Stylesheets erleichtert }
}

\newglossaryentry{jwt}
{
  name = JWT,
  description = {Jason Web Token. JSON Web Tokens sind eine offene Industriestandard RFC 7519 Methode zur sicheren Representierung von Forderungen zwischen zwei Parteien }
}

\newglossaryentry{promise}
{
  name = Promise,
  description = {Eine Repräsentation einer eventuellen Ausführung einer asynchronen Operation}

}

\makeglossaries
\usepackage[xindy]{imakeidx}
\makeindex


%------------------------------------------------------------------------------------------------------------------
%------ Eigenstaendigkeitserklaerung im gerahmten Kasten (parbox in einer framebox) ------
%------------------------------------------------------------------------------------------------------------------

\newcommand{\eigen}{
\setlength{\fboxsep}{2ex}
\setlength{\fboxrule}{0.8pt}
% Einstellungen fuer Rahmenabstand und Rahmendicke der Framebox
\begin{center}
  \fbox{
    \parbox{0.8\linewidth}{
    Ich versichere, die vorliegende Arbeit selbstst\"andig ohne fremde Hilfe verfasst
    und keine anderen Quellen und Hilfsmittel als die angegebenen benutzt zu haben.
    Die aus anderen Werken w\"ortlich entnommenen Stellen oder dem Sinn nach
    entlehnten Passagen sind durch Quellenangaben eindeutig kenntlich gemacht.
    \par\bigskip\bigskip\bigskip\bigskip
    \hspace*{0.8cm}Ort, Datum \hfill \vorname~\nachname\hspace*{0.8cm}
    }
  }
\end{center}
}

%%%%%%%%%%%%%%%%%%%%%%%%%%%%%%%%%%%%%%%%%%%%%%%%%

%\usepackage[applemac]{inputenc} % Inputencoding für Mac
%\usepackage[latin1]{inputenc} % Inputencoding für PC/Win
\usepackage[utf8]{inputenc} % Inputencoding, universell
%\usepackage[utf8x]{inputenc} % Inputencoding, universell


%------------------------ Titelblatt-Layout laden ----------------------------------

\input{hawmt-bachelor-titelblatt}
%\input{hawmt-master-titelblatt}

%---------------------------- Titeldefinitionen --------------------------------------

\newcommand{\vorname}{Maria}
\newcommand{\nachname}{Mustermann}
\newcommand{\matrikelnummer}{1234567}

\newcommand{\titel}{Titel Titel Titel Titel\\[0.2ex]
				\Large Untertitel Unter btitel Untertitel Untertitel}

\newcommand{\erstpruef}{Prof. Vorname Nachname}
\newcommand{\zweitpruef}{Prof. Vorname Nachname}

%\date{vorläufige Fassung vom \today}   % praktisch für Vorab-Versionen.
\date{\sffamily Hamburg, 2. 2. 2020}  % Abgabedatum!

%--------------------------------------------------------------------------------------
%----------------------------- hier gehts los! --------------------------------------
%--------------------------------------------------------------------------------------

\begin{document}
\selectlanguage{ngerman}
\maketitle           % Titelseite erzeugen
\newpage
\tableofcontents % Inhaltsverzeichnis erzeugen
\newpage


%------------ Zusammenfassung / Abstract ------------------

\thispagestyle{empty}
\selectlanguage{english}
\section*{\centering\abstractname}
Form and layout of this \LaTeX-template incorporate the guidelines for theses in the Media Technology Department \glqq Richtlinien zur Erstellung schriftlicher Arbeiten, vorrangig Bachelor-Thesis (BA) und Master-Thesis (MA) im Department Medientechnik in der Fa\-kul\-tät DMI an der HAW Hamburg\grqq\ in the version of December 6, 2012 by Prof.\ Wolfgang Willaschek.

The thesis should be printed single-sided (simplex). The binding correction (loss at the left aper edge due to binding) might be adjusted, according to the type of binding. This template incorporates a binding correction as BCOR=1mm (suitable for adhesive binding) in the \LaTeX\ document header.

{\bfseries This is the english version of the opening abstract} (don't forget to set \LaTeX's language setting back to ngerman after the english text).


\selectlanguage{ngerman}
\section*{\centering\abstractname}
Diese \LaTeX-Vorlage berücksichtigt in Form und Layout die Vorgaben für Abschlussarbeiten im Department Medientechnik \glqq Richtlinien zur Erstellung schriftlicher Arbeiten, vorrangig Bachelor-Thesis (BA) und Master-Thesis (MA) im Department Medientechnik in der Fakultät DMI an der HAW Hamburg\grqq, Fassung vom 6. Dezember 2012 von Prof. Wolfgang Willaschek.

Der Ausdruck soll einseitig erfolgen (Simplex). Je nach Bindung ist ggf. die Bindekorrektur (Verlust am linken Seitenrand durch die Bindung) noch anzupassen. In dieser Vorlage ist eine Bindekorrektur im header der \LaTeX-Datei mit BCOR=1mm für Klebebindung eingestellt.

{\bfseries Das ist die deutsche Version der vorangestellten Zusammenfassung. Beide Versionen -- englisch und deutsch -- sind verbindlich!}

\newpage

%--------------------------- Text -------------------------------

\chapter{Einleitung}

\section{Motivation}

 ``There is no cloud it's just someone else's Computer'' - eigentlich ein ganz
 triviales Statement, doch es wurde zu einem Internetphänomen, auch ``Meme''
 genannt, weil die Webindustrie es geschafft hat, für einen Benutzer
 transparent werden zu lassen, dass hinter so manchem Dienst sich in der Realität ein ganzes Rechenzentrum befindet.


Die große Rechenpower, die von jedem Ort, jeder Zeit verfügbar ist, machte auch
eine Breite verschiedener portabler Anzeigegeräte ubiquitär. Die Werbemarketing
Spezialisten sprechen von einem ``Second Screen'', aber in Wirklichkeit ist
jedes andere internetfähige Gerät gemeint, welches parallel zum laufenden
Fernsehprogramm genutzt wird und bei so manchem Anwender ist die Zahl
längst über zwei.


Viele kleine Applikationen sollen diese portablen Geräte zu intelligenten
persönlichen Assistenten machen, jedoch ist immer noch das beliebteste Programm
der Webbrowser. (App fatigue Artikel zitieren)


Ebenfalls hat sich an den Grundprotokollen, die das World Wide Web seit 1991 zu
Nutze macht nicht viel geändert. Obwohl ganz


%%% Local Variables:
%%% TeX-master: "../master"
%%% End:


\appendix
\chapter{Material}

\section{Fragebögen, Messprotokolle etc.}
In den Anhängen landen ggf. Listings, Fragebögen, Datenblätter, Messprotokolle, Skizzen zu Versuchsaufbauten und ähnliches Material zur Arbeit. Im \LaTeX-Dokument leitet der Befehl appendix die Anhänge ein.



%--------------------- VERZEICHNISSE ----------------

\listoffigures % Abbildungsverzeichnis erzeugen
\listoftables % Tabellenverzeichnis erzeugen

%--------------------- LITERATURLISTE ---------------
% Die Einträge sollen alphabetisch sortiert sein.

\begin{thebibliography}{}

% Formatierung für Internetquelle
% Grundregel: Name, Vorname (falls vorhanden), Vö-Jahr (falls vorhanden), Titel in Anführungszeichen, URL, Datum des letzten Aufrufs
% zur Formatierung der URL unbedingt den url-Befehl benutzen!!!
\bibitem[Blu-ray Disc Association(2005)]{bluray}
Blu-ray Disc Association:
\emph{White paper Blu-ray Disc Format 2.B Audio Visual Application, Format Specifications for BD-ROM},
\url{http://www.blu-raydisc.com/Assets/downloadablefile/2b_bdrom_audiovisualapplication_0305-12955-15269.pdf}, 2005, letzter Zugriff: 1. 10. 2012

% Formatierung für Aufsatz / Paper: Titel in Anführungszeichen, Zeitschriftentitel kursiv
\bibitem[Dooley \& Streicher(1982)]{dooley_streicher}
Dooley, Wesley L.  \& Streicher, Ronald D.:
\glqq M--S Stereo: A Powerful Technique for Working in Stereo\grqq,
\emph{Journ. Audio Engineering Society} vol. 30 (10), 1982

% Formatierung für Fachbuch, Diplomarbeit o.Ä.: Titel kursiv
\bibitem[Kuttruff(1991)]{kuttruff}
Kuttruff, Heinrich:
\emph{Room Acoustics}, 3. Aufl., Elsevier 1991

% Formatierung für Fachbuch mit Herausgeber und mehreren Autoren
\bibitem[Spehr(2009)]{spehr}
Spehr, Georg (Hrsg.):
\emph{Funktionale Klänge}, transcript 2009

% Formatierung für ein einzelnes Kapitel eines speziellen Autors aus einem Fachbuch mit mehreren Autoren
\bibitem[Sowodniok(2009)]{sowodniok}
Sowodniok, Ulrike:
\glqq Funktionaler Stimmklang -- Ein Prozess mit Nachhalligkeit\grqq,
in: Spehr, Georg (Hrsg.): \emph{Funktionale Klänge}, transcript 2009

% Formatierung für Aufsatz / Paper: Titel in Anführungszeichen, Zeitschriftentitel kursiv
\bibitem[Stephenson(1990)]{stephenson}
Stephenson, Uwe:
\glqq Comparison of the Mirror Image Source Method and the Sound Particle Simulation Method\grqq,
\emph{Applied Acoustics} vol. 29, 1990


\end{thebibliography}

%--------------------- EIGENSTÄNDIGKEITSERKLÄRUNG ---------------
\clearpage\thispagestyle{empty}
\eigen  % im header definiert
%--------------------------------------- ENDE ------------------------------------
\end{document}
%%%%%%%%%%%%%%%%%%%%%%%%%%%%%%%%%%%%











