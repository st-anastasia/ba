\newglossaryentry{databinding}
{
  name = Databinding,
  description = {  Eine Teschnik, welche die Benutzeroberfläche der App mit den dort angezeigten Daten verbindet und synchronisiert }
}

\newglossaryentry{json}
{
  name = JSON,
  description = { Javascript Object Notation. \url{http://json.org/} }
}

\newglossaryentry{di}
{
  name = Dependency Injection,
  description = { Ein Entwurfsmuster der objektorientierten Programmierung, bei dem Abhängigkeit zwischen Objekten erst zur Laufzeit hergestellt werden }
}

\newglossaryentry{ng-directive}
{
  name = AngularJS Direktive,
  description = { Marker auf dem DOM Element, welcher dem AngularJS Renderer mitteilt spezifisches Verhalten an das DOM Element zu binden. \url{https://docs.angularjs.org/guide/directive}}
}

\newglossaryentry{ng-service}
{
  name = Angular Service,
  description = { Ersetzbare Objekte, welche zusammengebunden werden um Code zu organisieren und zwischen Modulen zu teilen. \url{https://docs.angularjs.org/guide/services}}
}

\newglossaryentry{ng-factory}
{
  name = Angular Factory,
  description = { Komponenten, welche Services erzeugen. \url{https://docs.angularjs.org/guide/providers\#factory-recipe}}
}


\newglossaryentry{pug}
{
  name = Pug,
  description = { Javascript HTML Template Engine. \url{https://pugjs.org/} }
}

\newglossaryentry{transpiler}
{
  name = Transpiler,
  description = {Source to Source Compiler}
}

\newglossaryentry{localstorage}
{
  name = LocalStorage,
  description = { HTML5 Web Storage Schnittstelle. \url{https://www.w3schools.com/html/html5\_webstorage.asp} }
}


\newglossaryentry{webpack}
{
  name = Webpack,
  description = { Javascript module bundler. \url{https://webpack.js.org/}}
}

\newglossaryentry{es2016}
{
  name = ES2016,
  description = { Spezifische Version des ECMAScript standardisierten Sprachkerns von JavaScript. \url{http://www.ecma-international.org/ecma-262/7.0/}}
}

\newglossaryentry{sass}{
  name = SASS,
  description = { Stylesheet-Sprache, die als CSS-Präprozessor, mit Variablen, Schleifen und vielen anderen Funktionen, die Cascading Style Sheets (CSS) nicht mitbringen, die Erstellung von CSS vereinfacht und die Pflege großer Stylesheets erleichtert }
}

\newglossaryentry{jwt}
{
  name = JWT,
  description = {Jason Web Token. JSON Web Tokens sind eine offene Industriestandard RFC 7519 Methode zur sicheren Representierung von Forderungen zwischen zwei Parteien }
}

\newglossaryentry{promise}
{
  name = Promise,
  description = {Eine Repräsentation einer eventuellen Ausführung einer asynchronen Operation}

}
