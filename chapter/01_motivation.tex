\chapter{Einleitung}

\section{Motivation}
\label{sec:motivation}

 \enquote{There is no cloud it's just someone else's Computer} - eigentlich ein ganz triviales Statement, doch es wurde zu einem Internetphänomen, auch
 \enquote{Meme} genannt, weil die Internetindustrie es geschafft hat, für einen Benutzer transparent werden zu lassen, dass hinter so manchem Dienst sich in der Realität ein ganzes Rechenzentrum befindet.

Die große Rechenleistung, die von jedem Ort, jeder Zeit verfügbar ist, machte auch eine Breite verschiedener portabler Anzeigegeräte ubiquitär. Die Werbemarketingspezialisten sprechen von einem \enquote{Second Screen}, aber in Wirklichkeit ist jedes andere internetfähige Gerät gemeint, welches parallel zum laufenden Fernsehprogramm genutzt wird. Und bei der Auswahl aus Netbook, Tablet, Phablet, Smartphone, Smartwatch ist bei manchem Anwender die Zahl dieser Geräte längst über zwei. Viele kleine Applikationen sollen diese portablen Geräte zu intelligenten persönlichen Assistenten machen. (Techcrunch) spricht sogar von einem neuen Software Goldrausch, der in den letzten 7 Jahren stattgefunden hat.

\enquote{Like all gold rushes, they must come to an end. It is clear that everyone close to technology is suffering from what the market is calling “app fatigue.” If it wasn’t already hard enough to differentiate your app from the millions of others in the app store, it’s now becoming even harder. From a consumer perspective, there are just too many apps. New apps, by in large, are not providing nearly enough value for consumers to come back, and most simply replicate existing experiences with a story of a better design. Apps are not an order of magnitude better than their predecessor; thus, adoption drops off as quickly as it started.}\cite{Schippers:2006}

Es stellt sich daher eine Situation dar, in der die Anwender zwar die schnelle Reaktionsfähigkeit nativer Applikationen zu schätzen wissen, jedoch nicht sofort bereit sind weitere Software auf ihre Geräte zu installieren. Und das macht eine bestimmte Applikation wieder populärer den je - den Webrowser.

Auch im Zeitalter der Smartphone hat sich an den Grundprinzipien und Protokollen, die das World Wide Web seit 1991 zu Nutze macht, nicht viel geändert. Eine sog. Rich Internet Applikation wird vom Browser auf die gleiche Art und Weise geladen, wie die allererste HTML Webseite. Lediglich eine grundlegende Neuerung kam im Jahr 1995 hinzu. Netscape ermöglichte es den Entwicklern mehr Interaktivität durch Auslieferung vom Script Code in die statischen Webseiten einzubauen und leitete den Aufstieg von JavaScript - gegenwärtig einer der populärsten Programmiersprachen. Allerdings ist diese Berühmtheit ganz und gar nicht dadurch entstanden, dass die Sprache eine besonders elegante Erfindung war. Es war einfach die einzige von Browsern von Haus aus unterstützte Option. Ganz im Gegenteil, JavaScript basiert auf einigen schlechten Entscheidungen. Ein populäres Fachbuch nennt sich auch daher nicht umsonst - \"JavaScript - the good parts\".

\enquote{JavaScript is a language with more than its share of bad parts. It went from nonexistence to global adoption in an alarmingly short period of time. It never had an interval in the lab when it could be tried out and polished. It went straight into Netscape Navigator 2 just as it was, and it was very rough. When Java™ applets failed, JavaScript became the “Language of the Web” by default. JavaScript’s popularity is almost completely independent of its qualities as a programming language.

Fortunately, JavaScript has some extraordinarily good parts. In JavaScript, there is a beautiful, elegant, highly expressive language that is buried under a steaming pile of good intentions and blunders. The best nature of JavaScript is so effectively hidden that for many years the prevailing opinion of JavaScript was that it was an unsightly,
incompetent toy.}\cite[S. 2]{Crockford:2008}

Bei vielen Webanwendungen beschränkt sich daher heutzutage die Hauptinterkation immer noch darauf, beim Betätigen eines Knopfes neuen Markup vom Server zu laden. Alles, was an Benutzerinteraktion darüber hinaus geht, ist ein Nebengedanke. Und so kommt es vor, dass der serverseitige Teil, zuständig für das Rendern der Hauptinhalte, mit allen bekannten Prinzipien des guten Software Designs realisiert ist, der clientseitige, für die weitergehende Interaktivität zuständige JavaScript-Teil, aber eine bloße Ansammlung loser Scripte darstellt. Bei kleinem Anteil solchen clientseitigen Programmcodes wird diese Praxis aus Kostengründen toleriert, ist jedoch bei jeder mittleren Komplexität nicht mehr hinnehmbar. Diese Arbeit betrachtet die Implementierung eines solchen komplexen Webandwendung Clients.

\newpage

\section{Zielsetzung}
\label{sec:zielsetzung}


Diese Arbeit beschäftigt sich mit der Realisierung eines potentiell an Benutzerinteraktionen reichen, nativ ähnlichen Clients einer Webanwendung am Beispiel der Implementierung einer Photoverwaltungssoftware.

Dabei wird auf folgende Schwerpunkte eingegangen:
\begin{itemize}
\item Adaptierung an verschiedene Endgeräte
\item Auslagerung der Darstellungslogik an den Client
\item Architektonische Trennung von Verantwortlichkeiten
\item Authentifizierung der Benutzersitzung
\item Kommunikation mit dem Server
\end{itemize}


\section{Aufbau}

\newpage
