\chapter{Einleitung}

\section{Motivation}

 ``There is no cloud it's just someone else's Computer'' - eigentlich ein ganz
 triviales Statement, doch es wurde zu einem Internetphänomen, auch ``Meme''
 genannt, weil die Webindustrie es geschafft hat, für einen Benutzer
 transparent werden zu lassen, dass hinter so manchem Dienst sich in der Realität ein ganzes Rechenzentrum befindet.


Die große Rechenpower, die von jedem Ort, jeder Zeit verfügbar ist, machte auch
eine Breite verschiedener portabler Anzeigegeräte ubiquitär. Die Werbemarketing
Spezialisten sprechen von einem ``Second Screen'', aber in Wirklichkeit ist
jedes andere internetfähige Gerät gemeint, welches parallel zum laufenden
Fernsehprogramm genutzt wird und bei so manchem Anwender ist die Zahl
längst über zwei.


Viele kleine Applikationen sollen diese portablen Geräte zu intelligenten
persönlichen Assistenten machen, jedoch ist immer noch das beliebteste Programm
der Webbrowser. (App fatigue Artikel zitieren)


Ebenfalls hat sich an den Grundprotokollen, die das World Wide Web seit 1991 zu
Nutze macht nicht viel geändert. Obwohl ganz


%%% Local Variables:
%%% TeX-master: "../master"
%%% End: