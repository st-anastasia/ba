\chapter{Implementierung}

\section{Einleitung}

In 1 wurde bestimmt, dass diese Arbeit die Implementierung einer komplexen clientseitigen Webanwendung betrachtet. Ferner hat 3.4 gezeigt, dass hierfür eine komplette \textit{Ajax-Engine} an den Client über HTTP ausgeliefert werden muss.

\section{Grund Setup}
\label{sec:grund_setup}

Um die Ajax Engine zu bündeln, kommt \gls{webpack} zum Einsatz. Dieses erlaubt das Zusammenbauen mehrerer öffentlicher (vendor) und eigener JavaScript Module zu wenigen Buildfiles, welche bequem im Hauptmarkup eingebunden werden. Zudem wird dadurch ermöglicht, moderne JavaScript Syntax (e.g \textit{ES2016}) durch Code Preprocessing durch sog. Transpiler, zu verwenden. Ein solcher Transpiler wandelt z.B. einen modernen JavaScript Standard in einen ursprünglichen Standard, welcher von den Zielbrowsern der Anwendung bereits implementiert wurde, um.

Das Resultat eines solchen Builds ist eine einzige HTML Seite \textit{index.html}, die der Benutzer am Ende ausgeliefert bekommt. Die Index Seite bettet wiederum nur zwei zusammengebaute Scripte aus allen bestehenden ein - \textit{index.bundle.css} und \textit{index.bundle.js}. Diese sind entsprechend für das gesamte Layout und die gesamte Business Logik der Anwendung verantwortlich (siehe \ref{lst:index_html}).

\begin{listing}[H]
\begin{minted}{html}
<!doctype html>

<head>
  <title>Pika</title>
</head>
<body ng-app="pika">
...

<script src="build/vendor.bundle.js"></script>
<script src="build/index.bundle.js"></script>
</body>
</html>

\end{minted}
\caption{index.html}
\label{lst:index_html}
\end{listing}

\gls{webpack} setzt auf das in \textit{ES2015} eingeführte Modules Feature auf. 
Hierbei wird ein Entry Skript in der \textit{Webpack} Konfigurationsdatei deklariert. (ähnlich dem Main File eines herkömmlichen Programms). \textit{Webpack} durchläuft beim Build jedes darin importierte File rekursiv und lässt es anhand weiterer deklarierter Regeln durch entsprechende Präprozessoren verarbeiten. Sowohl Stylessheets als auch Skripte werden so verarbeitet. Der Output dieser Präprozessoren wird am Ende, wie in Listing \ref{lst:index_html} abgebildet, zu einzelnen Files gebündelt.

In Listing \ref{lst:webpack_config} ist ein Auszug aus der \textit{Pika} Builld Konfiguration zum Transpilieren von  \textit{ES2016} JavaScript Standard zu \textit{ES5} Standard mit Hilfe des \textit{babel-loader}, sowie von  \textit{SASS} zu \textit{CSS} mit Hilfe des \textit{sass-loader} dargestellt.

\begin{listing}[H]
\begin{minted}{javascript}
{
  entry: {
    index: ['./index.js']
  },
  module: {
    rules: [
      {test: /\.scss$/, use: ['sass-loader']},
      {test: /\.js$/, use: ['babel-loader']}
    ]
  }
}
\end{minted}
\caption{webpack.config.js}
\label{lst:webpack_config}
\end{listing}

Für diese Applikation werden insgesamt folgende Regeln deklariert:

\begin{itemize}
 \item Herkömmliche CSS Bündelung aus anderen Paketen
 \item Statische Assets Einbindung und Fingerprinting für besseres Browsercaching
 \item Präprozessing von \textit{ES2016} zu \textit{ES5} für moderne JavaScript Features, e.g. Klassen, Modularisierung, Arrow Functions, Parameter Destructuring, Promises   
 \item Präprozessing von \textit{SASS} zu \textit{CSS} für erweitertes Stylesheet Tooling, wie Schachtelung, Vererbung, Variablen Verwendung
 \item Präprozessing durch \textit{Pugs} (ehemal. \textit{Jade}) templates zu HTML für bessere Markup Lesbarkeit
 \item Extrahierung des separaten \textit{CSS} bundles
 \item Trennung von Vendor und Applikations Code in einzelne Bundles
\end{itemize}

Wie bereits oben erwähnt bekommt der Benutzer eine einzelne Seite mit den gebündelten Skripten ausgeliefert. Der tatsächliche Inhalt der Applikation wird also erst durch das Bundle Skript auf dem Client erzeugt. Hierfür muss das AngularJS Framework wissen, welcher Platzhalter im DOM das Ziel des clientseitigen Renderings darstellen soll.  

Listing \ref{lst:index_html} zeigt, dass der \textit{body} Tag hierfür mit dem enstprechenden Attribut \textit{ng-app="pika"} markiert wordern ist. Man spricht hier auch von einer \gls{ng-directive}. Damit AngularJS die Kontrolle über den Inhlalt des \textit{body} Tags übernehmen kann, ist schlußendlich der folgednde Afruf im \textit{index.js} File notwendig. 

\begin{listing}[H]
\begin{minted}{javascript}
  import angular from 'angular';
  angular.module('pika', [ 
    //list of submodules 
  ])
\end{minted}
\end{listing}


\section{Projekt Struktur}

In zahlreichen Einstiegs Quellen für Organisation von Angular Projekten, findet man eine Bündelung nach Funktionsverantwortung der Komponenten im Angular Framework vor. So werden etwa Controller, Direktiven, Factories, Services in einzelnen Unterverzeichnissen organisiert. Es ergibt sich etwa folgender Aufbau:

TODO: Aufbau

Dieser Aufbau hat den Nachteil, dass nicht sofort ersichtlich wird, wo sich der Applikationscode einer bestimmten Domain befindet, da er nun auf verschieden Orte verteilt ist. 

Pika verwendet daher eine Gliederung der Verzeichnisse nach ihrer Domain. Diese Praxis wird von der Angular Community favorisiert \footnote{https://stackoverflow.com/questions/18542353/angularjs-folder-structure} und etwa in \cite{Kukic:2014} detailliert dargestellt.

Die Analyse der Aufgabestellung für diese Anwendung im Kapitel Design ergab die Hauptdomainbereiche - Menünavigation (\textit{side-menu}), Authentifizierung und Session Handling (\textit{session}), Foto Gallerie und Suche (\textit{photos}), Detail Fotoansicht und Metadateneingabe (\textit{photo-detail}).
Es ergibt sich daher die Verzeichnis Struktur aus Listing \ref{lst:directorty_structure}.

\begin{listing}[H]
\begin{minted}{bash}
# Detail Fotoansicht und Metadateneingabe 
|-- photo-detail 
|   |-- controller.js
|   |-- form-component.js
|   |-- form.jade
|   |-- index.jade
|   |-- index.js
|   |-- index.scss
|   |-- toolbar-component.js
|   `-- toolbar.jade

# Foto Gallery und Suche
|-- photos
|   |-- client.js
|   |-- controller.js
|   |-- gallery.js
|   |-- index.jade
|   |-- index.js
|   |-- index.scss
|   |-- toolbar-component.js
|   `-- toolbar.jade

# Authentifizierung und Session Handling
|-- session
|   |-- controller.js
|   |-- index.js
|   `-- session.js

# Menünavigation
|-- side-menu
|   |-- index.jade
|   |-- index.js
|   `-- index.scss
|-- http-interceptor.js

# Übergreifender Code
|-- index.js
|-- index.scss
|-- routes.js
`-- theming.js
\end{minted}
\caption{Directory Structure}
\label{lst:directorty_structure}
\end{listing}

AngularJS erlaubt eine Modularisierung vorzunehmen um etwa Konflikte im Dependency Injection Container zu vermeiden. Entsprechend der Domain Aufteilung lässt sich also jeder einzelne Bereich als ein einzelenes Angular Modul abbilden.

Hierbei exportiert jedes einzelne Module seine eigene Angular API Deklaration (siehe \ref{lst:module_export}). Dise Exports werden schließlich im dem \textit{index.js} File an das \textit{Pika} Hauptmodul übergeben ( siehe \ref{lst:module_build}).


\begin{listing}[H]
\begin{minted}{javascript}
//photos/index.js
import angular from 'angular';

import client from './client';
import gallery from './gallery';
import controller from './controller';
import toolbarComponent from './toolbar-component.js';

export default angular.module('pika.photos', [
  client.name,
  gallery.name,
  controller.name,
  toolbarComponent.name
]);
 
\end{minted}
\caption{Modul Export}
\label{lst:module_export}
\end{listing}

\begin{listing}[H]
\begin{minted}{javascript}
//index.js
import session from './session';
import sideMenu from './side-menu';
import photos from './photos';
import photoDetail from './photo-detail';

angular.module('pika', [
  'ui.router',
  'ngMaterial',
  'ngFileUpload',
  sideMenu.name,
  session.name,
  photos.name,
  photoDetail.name
])
\end{minted}
\caption{Modul Zusammenbau}
\label{lst:module_build}
\end{listing}


\section{Layout}

\section{Routing}

Da Pika immer noch eine Webanwendung ist und im Browser ausgeführt wird, erwarten Benutzer, dass das Navigieren zwischen einzelnen Bereichen der Applikation mit einer Veränderung der URL in der Adresszeile des Browsers einhergeht. Ebenfalls soll es möglich sein zu einem bestimmten Bereich zu gelangen, indem man eine URL direkt in die Adresszeile des Browsers eingibt.

In \ref{sec:client_server} wurde jedoch erläutert, dass mit einer neuen Adresseingabe auch eine separate HTTP Anfrage verbunden ist. AngularJS bietet daher ein Routing Mechanismus an, welcher das Standardverhalten der Browser bei Adresseingaben kapert. 

URL Adresseingaben gehen zunächst durch den AngularJS Router in dem Client Code. Stellt dieser eine Adresse fest, welche in der Routingkonfiguration festgelegt wurde, so wird ein entsprechender Controller aufgerufen und seine Ausgabe in dem Hauptlayoutplatzhalter gerendert. 


\begin{listing}[H]
\begin{minted}{html}
<body ng-app="pika" layout="row" layout-fill>
<div layout="column" layout-fill ng-cloak="">
  <side-menu />
  <ui-view role="main" layout="column" layout-fill></ui-view>
</div>
</body>
\end{minted}
\caption{Hauptlayout}
\label{lst:main_layout}
\end{listing}

\ref{lst:main_layout} zeigt den restlichen Inhalt der \textit{index.html} Datei. Entscheidend ist die Direktive \textit{ng-app="pika"}. Sie sorgt dafür, dass das in \textit{index.js} \ref{lst:module_build} deklarierte Angular Hauptmodul das Rendering des Inhalts hier übernehmen kann.

Der Inhalt der Anwendung besteht aus den Direktiven für das Seitenmenü (\textit{side-menu} ) und dem jeweiligen Inhalt des Routers (\textit{ui-view}). 

Letztendlich brauch das AngularJS Hauptmodul eine Konfiguration für das Routing von URLs zu den jeweiligen Hauptkontrollern der Applikation. \ref{lst:routing_config} zeigt einen Auszug dieser  
Konfiguration. In dieser Arbeit wird ein 3rd Party Router, namens \textit{UI-Router} verwendet.
(Der Standardrouter ist bereits für den exemplarischen Andwendungsfall dieser Arbeit sehr eingeschränkt. Die Notwendigkeit für den UI-Router wird im weiteren Verlauf deutlich).

\begin{listing}[H]
\begin{minted}{javascript}
import photosTemplate from './photos/index.jade';
import photoDetailTemplate from './photo-detail/index.jade';

function routesConfig($stateProvider, $urlRouterProvider) {
  $stateProvider
    .state('photos', {
      url: '/photos/:page?search',
      template: photosTemplate,
      controller: 'photosController',
      controllerAs: '$ctrl',
      resolve: { authenticate: authenticate }
    })
    .state('photo-detail', {
      url: '/photo-detail/:id',
      template: photoDetailTemplate,
      controller: 'photoDetailController',
      controllerAs: '$ctrl',
      resolve: { authenticate: authenticate }
    });
}

\end{minted}
\caption{routes.js}
\label{lst:routing_config}
\end{listing}

Die Adresseingabe von \textit{https://pika.cloud/photos?page=7} führt entsprechend zum Instanziieren des \textit{PhotosController}. Dieser rendert die Seite 7 der Photogalerie in den \textit{ui-view}. Ein Klick auf den Link hinter der Miniansicht eines Bildes ruft die URL \textit{https://pika.cloud/photo-details/p13tr3} auf sorgt entsprechend, dass der \textit{PhotoDetailController} die Großansicht des Photos mit der ID  \textit{p13tr3} rendert. 

Zusät

\section{Session Handling}

\subsection{Grundverfahren}

Kapitel \ref{sec:authentication} definiert die Anforderung zur Benutzerauthentifizierung. Die Grundvorgehensweise dabei schildert sich wie folgt:

\begin{itemize}
  \item Der Benutzer gibt seine Anmeldedaten beim initialen Aufruf der Anwendung ein.
  \item Die Anmeldedaten werden auf dem Server validiert.
  \item Nach positiver Prüfung erhält der Benutzer einen Session Token.
  \item Der Client speichert den Session Token permanent ab.
  \item Der Session Token wird mit jeder weiteren Anfrage an den Server gesendet. 
\end{itemize}

Das obige Vorgehen liefert den Grundsatz einer Lösung auf das in Kapitel \ref{sec:client_server} geschilderte Problem der Zustandsbehaftung in dem zustandslosen HTTP Protokoll.
Jede HTTP Anfrage erhält schließlich einen eindeutigen Session Token über den der Server den Benutzer authentifizieren kann. 

Für die konkrete Implementierung existiert eine Reihe von Verfahren. Die einfachste Variante nutzt den Standard Cookie Mechanismus von HTTP um den Session Token zu speichern. Wenn ein solcher Cookie vom Server gesetzt wurde, ist kein weiterer Client Code erforderlich um diesen mit nachfolgenden Anfragen zu senden. Folglich ist es auch die ursprüngliche und einzige Variante, um rein serverseitige Webanwendungen zu authentifizieren.

Die Cookie Authentifizierung bietet allerdings die meiste Angriffsoberfläche. Aus diesem Grund verwenden eigen implementierte Webclients in aller Regel einen separaten Mechanismus. Der Standard JWT taucht oft in Zusammenhang von Authentifizierung von Single Page Applikationen auf.
Tatsächlich spezifiziert JWT nur den Aufbau- und Verifizierungsmechanismus der Session Token. 
Für den tatsächlichen Authentifizierung Flow bietet es lediglich Richtlinien an. In einfacher Form benötigt der Client kein Wissen darüber welche Art von Token Standard auf dem Server verwendet wurde. Daher zeigt diese Arbeit die Implementierung eines Authentifizierungsverfahren, welches einer solchen Richtlinie aus dem JWT Standard folgt. 

\subsection{Authentifizierung}

Den Aufruf der Root URL bearbeitet die \textit{SessionController\#index} Methode und rendert das Logn Formular, wenn der Benutzer nicht eigeloggt ist.

Die \textit{SessionController\#create} Methode nimmt die Anmeldedaten aus dem Login Formular entgegen. Daraufhin sendet der Controller \textit{POST '/api/session-token'} mit den Benutzereingaben an den Server. Sollte die Eingabe falsch sein, wird die Fehlermeldung der Antwort in einer Instanzvariable des Controllers abgelegt und per Databinding im Formular gerendert. Die gültige Antwort des Servers enthält ein JSON Objekt mit den Daten des gegenwärtigen Users mit dem JWT Session Token. 

Diese Daten müssen nun permanent abgelegt werden. Dieese Aufgaben übernimmt das \textit{Session} Objekt. Der \textit{SessionController} leitet die Daten des gegenwärtigen Users an die \textit{Session\#authorize} methode weiter. Darauf legt das Session Objekt diese über den \gls{LocalStorage} Mechanismus des Browsers ab. 

Beim erneuten Instaziieren des Session Objekts liest dessen Konstruktor das Token aus dem LocalStorage. Wird der \textit{SessionController\#index} erneut aufgerufen, prüft der \textit{SessionController} über \textit{Session\#isAuthenticated}, ob ein \textit{User} Objekt bekannt ist und übergibt das Rendering an den \textit{PhotosController}. Der Benutzer ist somit eingeloggt udn sieht die Photogallerie.

Nach dem die erfolgreiche Authentifizierung des Users abgeschlossen ist, müssen einzelne Bereiche der Applikation nur für eingeloggte Nutzer authoriziert werden. Hierfür muss das Session Objekt für andere Komponenten im System verfügbar gemacht werden. Solche Abhängigkeiten werden in AngularJS über einen \gls{Dependency Injection} Mechanismus aufgelöst. Der DI Kontainer kümmert sich um die Instanziierung aller registrierten Objekte. Es kann hier festgelegt wrden ob eine Abhängigkeit nur ein Mal pro Modul als \gls{Angular Service} verfügbar ist, oder jedes Mal neu mit Hilfe einer \gls{Angular Factory} instanziiert wird, wenn sie gebraucht wird. 


\begin{listing}[H]
\begin{minted}{javascript}
//session/controller.js
class SessionController {
  /** @ngInject */
  constructor($state, $http, session) {
    //...
  }

  index() {
    if (this.session.isAuthenticated()) {
      return this.$state.go('photos')
    }
  }

  create() {
    const _this = this;
    this.$http.post('/api/session-token', this.user)
    .then((response) => {
      _this.session.auth(response.data);
      _this.$state.go('photos');
    });
  }
}

export default angular.module('session.controller', [])
  .controller('sessionController', SessionController)

//session/session.js
class Session {

  constructor() {
    this.user = JSON.parse(localStorage.getItem('user')) || {};
  }

  auth(user) {
    this.user = user;
    localStorage.setItem('user', JSON.stringify(user));
  }

  isAuthenticated() {
    return !!this.user.token;
  }
}

export default angular.module('session.session', [])
  .service('session', Session)

\end{minted}
\caption{Session Handling}
\label{lst:session_handling}
\end{listing}

\subsection{Authorizierung}

Die \textit{Session} Klasse wird als Service regestriert, da dessen Instanz mit den User Daten über alle anforderdenen Komponenten einmalig verfügbar sein muss. Eine weitere Stelle wo der authentifizierte Zustand geprüft wird, ist der Router. Alle Routes im System müssen den Benutzer zurück auf das Login Formular weiterleiten, wenn dieser nicht eingeloggt ist.


