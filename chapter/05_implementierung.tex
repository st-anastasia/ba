\chapter{Implementierung}

\section{Einleitung}

In 1 wurde bestimmt, dass diese Arbeit die Implementierung einer komplexen clientseitigen Webanwendung betrachtet. Ferner hat 3.4 gezeigt, dass hierfür eine komplette \textit{Ajax-Engine} an den Client über HTTP ausgeliefert werden muss.

\section{Grund Setup}
\label{sec:grund_setup}

Um die Ajax Engine zu bündeln, kommt \gls{webpack} zum Einsatz. Dieses erlaubt das Zusammenbauen mehrerer öffentlicher (vendor) und eigener JavaScript Module zu wenigen Buildfiles, welche bequem im Hauptmarkup eingebunden werden. Zudem wird dadurch ermöglicht, moderne JavaScript Syntax (e.g \textit{ES2016}) durch Code Preprocessing durch sog. Transpiler, zu verwenden. Ein solcher Transpiler wandelt z.B. einen modernen JavaScript Standard in einen ursprünglichen Standard, welcher von den Zielbrowsern der Anwendung bereits implementiert wurde, um.

Das Resultat eines solchen Builds ist eine einzige HTML Seite \textit{index.html}, die der Benutzer am Ende ausgeliefert bekommt. Die Index Seite bettet wiederum nur zwei zusammengebaute Scripte aus allen bestehenden ein - \textit{index.bundle.css} und \textit{index.bundle.js}. Diese sind entsprechend für das gesamte Layout und die gesamte Business Logik der Anwendung verantwortlich (siehe \ref{lst:index_html}).

\begin{listing}[H]
\begin{minted}{html}
<!doctype html>

<head>
  <title>Pika</title>
</head>
<body ng-app="pika">
...

<script src="build/vendor.bundle.js"></script>
<script src="build/index.bundle.js"></script>
</body>
</html>

\end{minted}
\caption{index.html}
\label{lst:index_html}
\end{listing}

\gls{webpack} setzt auf das in \textit{ES2015} eingeführte Modules Feature auf. 
Hierbei wird ein Entry Skript in der \textit{Webpack} Konfigurationsdatei deklariert. (ähnlich dem Main File eines herkömmlichen Programms). \textit{Webpack} durchläuft beim Build jedes darin importierte File rekursiv und lässt es anhand weiterer deklarierter Regeln durch entsprechende Präprozessoren verarbeiten. Sowohl Stylessheets als auch Skripte werden so verarbeitet. Der Output dieser Präprozessoren wird am Ende, wie in Listing \ref{lst:index_html} abgebildet, zu einzelnen Files gebündelt.

In Listing \ref{lst:webpack_config} ist ein Auszug aus der \textit{Pika} Builld Konfiguration zum Transpilieren von  \textit{ES2016} JavaScript Standard zu \textit{ES5} Standard mit Hilfe des \textit{babel-loader}, sowie von  \textit{SASS} zu \textit{CSS} mit Hilfe des \textit{sass-loader} dargestellt.

\begin{listing}[H]
\begin{minted}{javascript}
{
  entry: {
    index: ['./index.js']
  },
  module: {
    rules: [
      {test: /\.scss$/, use: ['sass-loader']},
      {test: /\.js$/, use: ['babel-loader']}
    ]
  }
}
\end{minted}
\caption{webpack.config.js}
\label{lst:webpack_config}
\end{listing}

Für diese Applikation werden insgesamt folgende Regeln deklariert:

\begin{itemize}
 \item Herkömmliche CSS Bündelung aus anderen Paketen
 \item Statische Assets Einbindung und Fingerprinting für besseres Browsercaching
 \item Präprozessing von \textit{ES2016} zu \textit{ES5} für moderne JavaScript Features, e.g. Klassen, Modularisierung, Arrow Functions, Parameter Destructuring, Promises   
 \item Präprozessing von \textit{SASS} zu \textit{CSS} für erweitertes Stylesheet Tooling, wie Schachtelung, Vererbung, Variablen Verwendung
 \item Präprozessing durch \textit{Pugs} (ehemal. \textit{Jade}) templates zu HTML für bessere Markup Lesbarkeit
 \item Extrahierung des separaten \textit{CSS} bundles
 \item Trennung von Vendor und Applikations Code in einzelne Bundles
\end{itemize}

Wie bereits oben erwähnt bekommt der Benutzer eine einzelne Seite mit den gebündelten Skripten ausgeliefert. Der tatsächliche Inhalt der Applikation wird also erst durch das Bundle Skript auf dem Client erzeugt. Hierfür muss das AngularJS Framework wissen, welcher Platzhalter im DOM das Ziel des clientseitigen Renderings darstellen soll.  

Listing \ref{lst:index_html} zeigt, dass der \textit{body} Tag hierfür mit dem enstprechenden Attribut \textit{ng-app="pika"} markiert wordern ist. Man spricht hier auch von einer \gls{ng-directive}. Damit AngularJS die Kontrolle über den Inhlalt des \textit{body} Tags übernehmen kann, ist schlußendlich der folgednde Afruf im \textit{index.js} File notwendig. 

\begin{listing}[H]
\begin{minted}{javascript}
  import angular from 'angular';
  angular.module('pika', [ 
    //list of submodules 
  ])
\end{minted}
\end{listing}

\section{Routing}
\label{sec:routing}

Da Pika immer noch eine Webanwendung ist und im Browser ausgeführt wird, erwarten Benutzer, dass das Navigieren zwischen einzelnen Bereichen der Applikation mit einer Veränderung der URL in der Adresszeile des Browsers einhergeht. Ebenfalls soll es möglich sein zu einem bestimmten Bereich zu gelangen, indem man eine URL direkt in die Adresszeile des Browsers eingibt.

In \ref{sec:client_server} wurde jedoch erläutert, dass mit einer neuen Adresseingabe auch eine separate HTTP Anfrage verbunden ist. AngularJS bietet daher ein Routing Mechanismus an, welcher das Standardverhalten der Browser bei Adresseingaben kapert. 

URL Adresseingaben gehen zunächst durch den AngularJS Router in dem Client Code. Stellt dieser eine Adresse fest, welche in der Routingkonfiguration festgelegt wurde, so wird ein entsprechender Controller aufgerufen und seine Ausgabe in dem Hauptlayoutplatzhalter gerendert. 


\begin{listing}[H]
\begin{minted}{html}
<body ng-app="pika" layout="row" layout-fill>
<div layout="column" layout-fill ng-cloak="">
  <side-menu />
  <ui-view role="main" layout="column" layout-fill></ui-view>
</div>
</body>
\end{minted}
\caption{Hauptlayout}
\label{lst:main_layout}
\end{listing}

Listing \ref{lst:main_layout} zeigt den restlichen Inhalt der \textit{index.html} Datei. Entscheidend ist die Direktive \textit{ng-app="pika"}. Sie sorgt dafür, dass das in \textit{index.js} (\ref{lst:module_build}) deklarierte Angular Hauptmodul das Rendering des Inhalts hier übernehmen kann.

Der Inhalt der Anwendung besteht aus den Direktiven für das Seitenmenü (\textit{side-menu} ) und dem jeweiligen Inhalt des Routers (\textit{ui-view}). 

Letztendlich braucht das AngularJS Hauptmodul eine Konfiguration für das Routing von URLs zu den jeweiligen Hauptkontrollern der Applikation. Listing \ref{lst:routing_config} zeigt einen Auszug dieser Konfiguration. In dieser Arbeit wird ein 3rd Party Router, namens \textit{UI-Router} verwendet. (Der Standardrouter ist bereits für den exemplarischen Anwendungsfall dieser Arbeit sehr eingeschränkt. Die Notwendigkeit für den UI-Router wird im weiteren Verlauf deutlich).

\begin{listing}[H]
\begin{minted}{javascript}
import photosTemplate from './photos/index.jade';
import photoDetailTemplate from './photo-detail/index.jade';

function routesConfig($stateProvider, $urlRouterProvider) {
  $stateProvider
    .state('photos', {
      url: '/photos/:page?search',
      template: photosTemplate,
      controller: 'photosController',
      controllerAs: '$ctrl',
      resolve: { authenticate: authenticate }
    })
    .state('photo-detail', {
      url: '/photo-detail/:id',
      template: photoDetailTemplate,
      controller: 'photoDetailController',
      controllerAs: '$ctrl',
      resolve: { authenticate: authenticate }
    });
}

\end{minted}
\caption{routes.js}
\label{lst:routing_config}
\end{listing}

Die obige Konfiguration definiert folgendes Verhalten:
Die Adresseingabe von \textit{https://pika.cloud/photos?page=7} führt zum Instanziieren des \textit{PhotosController}. Dieser rendert die Seite 7 der Photogalerie in den \textit{ui-view}. Ein Klick auf den Link hinter der Miniansicht eines Bildes ruft die URL \textit{https://pika.cloud/photo-details/p13tr3} auf und sorgt entsprechend dafür, dass der \textit{PhotoDetailController} die Großansicht des Photos mit der ID  \textit{p13tr3} rendert.

\section{Projekt Struktur}
\label{sec:project_structure}

Die Analyse der Aufgabestellung für diese Anwendung im Kapitel Design ergab die Hauptdomainbereiche -  Authentifizierung und Session Handling (\textit{session}), Menünavigation (\textit{side-menu}), Foto Gallerie und Suche (\textit{photos}), Detail Fotoansicht und Metadateneingabe (\textit{photo-detail}). Entprechend dieser Domainbereiche wird die Modularisierung vorgenommen.

\subsection{Verzeichnis Struktur}

In zahlreichen Einstiegs Quellen für Organisation von Angular Projekten, findet man eine Bündelung nach Funktionsverantwortung der Komponenten im Angular Framework vor. So werden etwa Controller, Direktiven, Factories, Services in einzelnen Unterverzeichnissen organisiert. Es ergibt sich etwa folgender Aufbau:

TODO: Aufbau

Dieser Aufbau hat den Nachteil, dass nicht sofort ersichtlich wird, wo sich der Applikationscode einer bestimmten Domain befindet, da er nun auf verschieden Orte verteilt ist. 

Pika verwendet daher eine Gliederung der Verzeichnisse nach ihrer Domain. Diese Praxis wird von der Angular Community favorisiert \footnote{https://stackoverflow.com/questions/18542353/angularjs-folder-structure} und etwa in \cite{Kukic:2014} detailliert dargestellt. Es ergibt sich daher die Verzeichnis Struktur aus Listing \ref{lst:directorty_structure}.


\begin{listing}[H]
\begin{minted}{bash}
# Detail Fotoansicht und Metadateneingabe 
|-- photo-detail 
|   |-- controller.js
|   |-- form-component.js
|   |-- form.jade
|   |-- index.jade
|   |-- index.js
|   |-- index.scss
|   |-- toolbar-component.js
|   `-- toolbar.jade

# Foto Gallery und Suche
|-- photos
|   |-- client.js
|   |-- controller.js
|   |-- gallery.js
|   |-- index.jade
|   |-- index.js
|   |-- index.scss
|   |-- toolbar-component.js
|   `-- toolbar.jade

# Authentifizierung und Session Handling
|-- session
|   |-- controller.js
|   |-- index.js
|   `-- session.js

# Menünavigation
|-- side-menu
|   |-- index.jade
|   |-- index.js
|   `-- index.scss
|-- http-interceptor.js

# Übergreifender Code
|-- index.js
|-- index.scss
|-- routes.js
`-- theming.js
\end{minted}
\caption{Directory Structure}
\label{lst:directorty_structure}
\end{listing}

\subsection{Modul Struktur}

AngularJS erlaubt eine Modularisierung vorzunehmen um etwa Konflikte im Dependency Injection Container zu vermeiden. Entsprechend der Domain Aufteilung lässt sich also jeder einzelne Bereich als ein einzelnes Angular Modul abbilden.

Hierbei exportiert jedes einzelne Module seine eigene Angular API Deklaration (siehe \ref{lst:module_export}). Diese Exports werden schließlich im dem \textit{index.js} File an das \textit{Pika} Hauptmodul übergeben ( siehe \ref{lst:module_build}).


\begin{listing}[H]
\begin{minted}{javascript}
//photos/index.js
import angular from 'angular';

import client from './client';
import gallery from './gallery';
import controller from './controller';
import toolbarComponent from './toolbar-component.js';

export default angular.module('pika.photos', [
  client.name,
  gallery.name,
  controller.name,
  toolbarComponent.name
]);
 
\end{minted}
\caption{Modul Export}
\label{lst:module_export}
\end{listing}

\begin{listing}[H]
\begin{minted}{javascript}
//index.js
import session from './session';
import sideMenu from './side-menu';
import photos from './photos';
import photoDetail from './photo-detail';

angular.module('pika', [
  'ui.router',
  'ngMaterial',
  'ngFileUpload',
  sideMenu.name,
  session.name,
  photos.name,
  photoDetail.name
])
\end{minted}
\caption{Modul Zusammenbau}
\label{lst:module_build}
\end{listing}

\subsection{Komponenten Struktur}

Kapitel \ref{} erläutert das Paradigma der clientseitigen Javascript MVC Frameworks. Es liegt darin einzelne Komponenten für Teilbereiche des HTML DOMs zu erstellen. Die Komponente nimmt alle Benutzerinteraktion in diesem Teilbereich entgegen, speichert und rendert dessen Zustand. 

Sie besteht aus mehreren weiteren Einheiten - einem View, einem Controller und einem bist mehreren Models. In dieser Arbeit nimmt der Controller, wie am Ende von  Kapitel \ref{} geschildert eine Presenter Rolle ein. D.h. er speichert nicht nur die Rohdaten für die Darstellung, sondern exakt den Zustand des Benutzerinterfaces. 

Betrachtet an dem Prozess der Photogalerie Darstellung bedeutet es beispielsweise, dass der entsprechende PhotosController die Daten exakt in der Form speichert wie diese betrachtet werden. Die Darstellung der Photogalerie ist nach Monat der Aufnahme gruppiert, die Rohdaten werden aber üblicherweise von eine Serverschnittselle als sortierte Liste zurückkommen. Es ist demnach die Aufgabe des Controllers in der Presenter Rolle, die Daten aus einer Liste gruppiert nach Aufnahme Monat abzulegen. Dafür kann er natürlich weitere Model Services hinzuziehen. Der View Markup bleibt somit sehr schlicht. Dieses Vorgehen ergibt eine Maximum aus Trennung von visueller Darstellung und Applikationslogik.

Komponenten in Angular können entweder durch URL Routing in einem Platzhalter ausgetauscht werden (siehe \ref{sec:routing}) oder direkt durch eine Angular Direktive als eine Art Custom HTML Tag instaziert werden, falls sich ihr Verhalten nicht auf die URL Änderung auswirken muss. 

Für diese Arbeit ergibt sich folgende generelle Komponenten Gliederung:

\begin{itemize}
\item Die vier Hauptdomainbereiche - session, side-menu, photos, photo-details - bilden entsprechende Komponenten und bestehen aus einem Hauptcontroller und, session ausgenommen, aus einem zugehörigen Markup Template in \gls{jade} syntax. 
\item Photos- und PhotoDetails Controller werden über URL Routing angesprochen
\item Die Hauptbereiche besitzen weitere Komponenten Gliederungen, wie PhotoToolbar, PhotoDetails Toolbar, PhotoForm. Sie werden als Direktiven im Markup initialisiert.
\end{itemize}

\section{Session Handling}

\subsection{Grundverfahren}
\label{sec:session_handling_basics}

Kapitel \ref{sec:authentication} definiert die Anforderung zur Benutzerauthentifizierung. Die Grundvorgehensweise dabei schildert sich wie folgt:

\begin{itemize}
  \item Der Benutzer gibt seine Anmeldedaten beim initialen Aufruf der Anwendung ein.
  \item Die Anmeldedaten werden auf dem Server validiert.
  \item Nach positiver Prüfung erhält der Benutzer einen Session Token.
  \item Der Client speichert den Session Token permanent ab.
  \item Der Session Token wird mit jeder weiteren Anfrage an den Server gesendet. 
\end{itemize}

Das obige Vorgehen liefert den Grundsatz einer Lösung auf das in Kapitel \ref{sec:client_server} geschilderte Problem der Zustandsbehaftung in dem zustandslosen HTTP Protokoll. Jede HTTP Anfrage erhält durch die obige Technik schließlich einen eindeutigen Session Token, über den der Server den Benutzer authentifizieren kann. 

Für die konkrete Implementierung existiert eine Reihe von Verfahren. Die einfachste Variante nutzt den Standard Cookie Mechanismus von HTTP um den Session Token zu speichern. Wenn ein solcher Cookie vom Server gesetzt wurde, ist kein weiterer Client Code erforderlich um diesen mit nachfolgenden Anfragen zu senden. Folglich ist es auch die ursprüngliche und einzige Variante, um rein serverseitige Webanwendungen zu authentifizieren.

Die Cookie Authentifizierung bietet allerdings die meiste Angriffsoberfläche. Aus diesem Grund verwenden eigen implementierte Webclients in aller Regel einen separaten Mechanismus. Der Standard JWT taucht oft in Zusammenhang von Authentifizierung von Single Page Applikationen auf.
Tatsächlich spezifiziert JWT nur den Aufbau- und Verifizierungsmechanismus der Session Token. 

Für den tatsächlichen Authentifizierung Flow bietet es lediglich Richtlinien an. In einfacher Form benötigt der Client kein Wissen darüber, welche Art von Token Standard auf dem Server verwendet wurde. Daher zeigt diese Arbeit die Implementierung eines Authentifizierungsverfahren, welches einer solchen Richtlinie aus dem JWT Standard folgt. 

\subsection{Authentifizierung}

Den Aufruf der Root URL bearbeitet die \textit{SessionController\#index} Methode und rendert das Login Formular, wenn der Benutzer nicht eingeloggt ist (siehe \ref{lst:session_handling}).

Die \textit{SessionController\#create} Methode nimmt die Anmeldedaten aus dem Login Formular entgegen. Daraufhin sendet der Controller \textit{POST '/api/session-token'} mit den Benutzereingaben an den Server. Sollte die Eingabe falsch sein, wird die Fehlermeldung der Antwort in einer Instanzvariable des Controllers abgelegt und per Databinding im Formular gerendert. Die gültige Antwort des Servers enthält ein JSON Objekt mit den Daten des gegenwärtigen Users inklusive des JWT Session Tokens. 

Diese Daten müssen nun permanent abgelegt werden. Diese Aufgaben übernimmt das \textit{Session} Objekt. Der \textit{SessionController} leitet die Daten des gegenwärtigen Users an die \textit{Session\#authorize} Methode weiter. Darauf legt das Session Objekt diese über den \gls{localstorage} Mechanismus des Browsers ab. 

Beim erneuten Instanziieren des Session Objekts liest dessen Konstruktor das Token aus dem LocalStorage. Wird der \textit{SessionController\#index} erneut aufgerufen, prüft der \textit{SessionController} über \textit{Session\#isAuthenticated}, ob ein \textit{User} Objekt bekannt ist und übergibt das Rendering an den \textit{PhotosController}. Dies wird über die Router Schnittstelle \textit{this.\$state.go('photos')} erreicht. Der Befehl verändert die Adresszeile des Browsers auf die unter dem  \textit{photos} Schlüssel konfigurierte URL. Ebenfalls unter diesem Schlüssel in der Routerkonfiguration
wird festgelegt, dass der \textit{PhotosController} für die Navigation auf die obige URL zuständig ist. Angular sorgt dafür, dass der entsprechende Controller instanziiert wird und das Rendering übernimmt. Der Benutzer ist somit eingeloggt und sieht die Photogalerie.

\begin{listing}[H]
\begin{minted}{javascript}
//session/controller.js
class SessionController {
  /** @ngInject */
  constructor($state, $http, session) {
    //...
  }

  index() {
    if (this.session.isAuthenticated()) {
      return this.$state.go('photos')
    }
  }

  create() {
    const _this = this;
    this.$http.post('/api/session-token', this.user)
    .then((response) => {
      _this.session.authenticate(response.data);
      _this.$state.go('photos');
    });
  }
}

export default angular.module('session.controller', [])
  .controller('sessionController', SessionController)

//session/session.js
class Session {

  constructor() {
    this.user = JSON.parse(localStorage.getItem('user')) || {};
  }

  authenticate(user) {
    this.user = user;
    localStorage.setItem('user', JSON.stringify(user));
  }

  isAuthenticated() {
    return !!this.user.token;
  }
}

export default angular.module('session.session', [])
  .service('session', Session)

\end{minted}
\caption{Session Handling}
\label{lst:session_handling}
\end{listing}

\subsection{Autorisierung}

Nach dem die erfolgreiche Authentifizierung des Users abgeschlossen ist, müssen einzelne Bereiche der Applikation nur für eingeloggte Nutzer autorisiert werden. Hierfür muss das Session Objekt für andere Komponenten im System verfügbar gemacht werden. 

Solche Abhängigkeiten werden in AngularJS über das \gls{di} Mechanismus aufgelöst. Der DI Con­tai­ner kümmert sich um die Instanziierung aller registrierten Objekte. Es kann hier festgelegt werden ob eine Abhängigkeit nur ein Mal pro Modul als \gls{ng-service} verfügbar ist, oder jedes Mal neu mit Hilfe einer \gls{ng-factory} instanziiert wird, wenn sie gebraucht wird. 
Die \textit{Session} Klasse wird als Service registriert, da dessen Instanz mit den User Daten über alle anfordernden Komponenten einmalig verfügbar sein muss. 

Eine Stelle, wo der authentifizierte Zustand geprüft wird, ist der Router. Alle Routes im System müssen den Benutzer zurück auf das Login Formular weiterleiten, wenn dieser nicht eingeloggt ist (siehe \ref{lst:route_authorization}).

\begin{listing}[H]
\begin{minted}{javascript}
//routes.js
function routesConfig($stateProvider, $urlRouterProvider) {
  $stateProvider
  // ...
    .state('photos', {
      url: '/photos/:page?search',
      template: photosTemplate,
      controller: 'photosController',
      controllerAs: '$ctrl',
      resolve: { authenticate: authenticate }
    })
  // ... 
}

function authenticate($q, $state, $timeout, session) {
  if (session.isAuthenticated()) {
    return $q.when();
  }

  $timeout(function() {
    $state.go('session');
  })

  return $q.reject();
}

\end{minted}
\caption{Route Autorisierung}
\label{lst:route_authorization}
\end{listing}

Über das \textit{resolve} keyword lässt sich in der Router Konfiguration eine \textit{authenticate} Funktion anhängen. Diese wird immer aufgerufen, wenn der eine Routeänderung auf die entsprechende Route passiert. Der Returnwert dieser Funktion - ein erfolgreich/nicht erfolgreich aufgelöstes Promise Objekt - bestimmt darüber, ob die Route Änderung wirklich durchgeführt werden darf.

Diese Entscheidung kann die \textit{authenticate} Funktion einfach treffen, indem es die API des injecteten Session Objektes \textit{session.isAuthenticated()} in Anspruch nimmt.

Das oben beschrieben Verfahren der Route Autorisierung ist offensichtlich keine Sicherheitsmaßnahme, da das Ganze in dem Client Code passiert. Es dient, wie angedeutet, lediglich der notwendigen User Experience - der User soll immer auf das Login Formular umgeleitet werden, falls nicht richtig autorisiert. 

Wie in Kapitel \ref{sec:session_handling_basics} geschildert, muss ein Server jeded HTTP Anfrage auf das valide Session Token prüfen, da diese auch losgelöst von dem Angular Client von einem potentiellen Angreifer durchgeführt werden können. Das Session Objekt muss demnach bei allen HTTP Aufrufen bekannt sein. 

Angular bringt einen HTTP Client mit, welcher lediglich einen bequemeren Wrapper über die native Browser \textit{fetch} API darstellt. Über einen sog. \textit{HTTP Interceptor} lässt sich das Setzen des nötigen Autorisierungheaders einmalig konfigurieren (siehe \ref{lst:http_interceptor}).

\begin{listing}[H]
\begin{minted}{javascript}

//http-interceptor.js
export default function (session) {
  return {
    request: (config) => {
      if (session.isAuthenticated()) {
        config.headers.Authorization = `Bearer ${session.user.token}`;
      }
      return config;
    }
  };
};

\end{minted}
\caption{HTTP Interceptor}
\label{lst:http_interceptor}
\end{listing}

Der HTTP Client lässt sich wie jede Abhängikeit in Angular über den DI Container übergeben. Nach der korrekten Registrierung des HTTP Interceptor muss bei der Benutzung des Client nichts weiter beachtet werden (Siehe \ref{lst:http_client}).

\begin{listing}[H]
\begin{minted}{javascript}

//photos-client.js
class PhotosClient {
  constructor($http {
    this.$http = $http;
  }

  // ...

  findById(id) {
    return this.$http.get(`/api/photos/${id}`);
  }

  // ...
}  

\end{minted}
\caption{HTTP Client Verwendung}
\label{lst:http_client}
\end{listing} 

\section{Menüführung}

In Kapitel .. wurde ein Seitenmenü für die Hauptnavigation der Applikation und ein jeweils ein Untermenü für die Unterbereiche, wie die Galerieansicht oder Photo Detail Ansicht entworfen. Das Haupt Seitenmenü bleibt während des gesamten Nutzererlbnis gleich, während die Untermenüs sich je nach Kontext ändern können.

In Listing \ref{lst:main_layout} wurde der Lösungsansatz deutlich. Die Applikation benötigt ein Haupt Layout bzw. ein übergreifendes Frame. Dieses Frame enthält das Seitenmenü, dessen Komponente über die Angular Direktive \textit{side-menu} im Markup initialisiert wird. Neben dem Seitenmenü ist ein Platzhalter für die Unterbereiche, dessen Inhalt von dem Router bestimmt wird. Die verschiedenen Kontext Untermenüs hingegen werden von der jeweiligen Hauptkomponente аls direktive in Markup eingebunden. Dies hat eben der Grund, dass der Kontext sich mit dem Wechsel in einen Unterbereich unterscheidet. Die Photogallerie benötigt beispielsweise ein Suchfeld, während das die Detailansicht ein Button für das Anzeigen der Metadaten enthält.

Die Implementierung des Seitenmenüs gestaltet sich recht trivial. Der SideMenuController (siehe Listing \ref{lst:side_menu_controller}) initialisiert Menüeinträge. Es sind Lediglich Einträge mit Namen und URL Route zu dem Unterbereich. Der Controller muss somit nicht Mal auf die Navigation selbst reagieren, da diese von dem Router übernommen wird.


\begin{listing}[H]
\begin{minted}{javascript}

import angular from 'angular';
import template from './index.jade';

class SideMenuController {

  /** @ngInject */
  constructor($mdSidenav) {
    this.$mdSidenav = $mdSidenav;
    this.initMenu();
    this.initAdminMenu();
  }

  initMenu() {
    this.menu = [
      {
        link: '/photos',
        title: 'Photos',
        icon: 'image',
      },
      {
        link: '/settings',
        title: 'Settings',
        icon: 'settings',
      },
      //...
    ];
  }

  close() {
    this.$mdSidenav('left').close();
  }
}

export default angular.module('pika.side-menu', [])
  .component('sideMenu', {
    template: template(),
    controller: SideMenuController,
    controllerAs: 'sideMenuController',
  });


\end{minted}
\caption{SideMenuController}
\label{lst:side_menu_controller}
\end{listing} 


\begin{listing}[H]

//side-menu/index.jade
md-sidenav.md-sidenav-left(/* ... */)
  // ...
  md-content
    md-list(flex='')
      md-list-item(ng-repeat='item in sideMenuController.menu')
        md-icon.blue_grey-400.material-icons.step {{item.icon}}
        p a(href='item.link')
            {{item.title}}
     
    md-button(ng-click='sideMenuController.close()', class='md-primary') Close

\caption{Side Menu View}
\label{lst:side_menu_view}
\end{listing} 

Das Verhalten des Seitenmenüs inklusive Ein- und Ausklappen Animation bringt die Angular \texit{mdSideNav} Komponente mit, welche schließlich im Markup erzeugt wird (siehe Listing \ref{lst:side_menu_view}). Paradoxerweise ist der SideMenuController nicht für das ausklappen des Seitenmenüs zuständig. Diese Aufgabe übernehmen die Kontextmenüs. Dies hat den Grund, dass hat den einfachen Grund, dass der Ausklappregler sich in dem Kontextmenü links. Der Button zum ausklappen lässt sich ebenfalls eine extra Komponent auslagern und in jede Kontext Toolbar integrieren. 

\section{Photo Galerie}

Die Photogalerie ist ein Exempel für die Kommunikation über mehrere Controller und Wiederverwendbarkeit von Models. Denn die Einzelansicht des Photos ist ebenfalls ein Art Galerie, da hier das Sliden zum nächsten und vorherigen Photo ermöglicht werden soll. Aus diesem Grund entsteht die Notwendigkeit das Laden von Photos in Chunks an ein separates Model - den \texit{PhotosService} auszulagern. 

Der \texit{PhotosService} kann zwischen den \textit{PhotosController} und \textit{PhotoDetailsController} wiederverwendet werden. Er speichert den aktuell geladenen Chunk an Photos sowie das aktuell ausgewählte Photo. Navigiert der Benutzer auf die nächste Galerieseite oder will weiter als den letzte gecachte State in dem Service sliden, wird der nächste Chunk geladen. Die Controller und Views beobachten diese Änderung und stellen den neuen Zustand dar.



Die Navigation zu \textit{/photos} rendert die Photo Galerie. Dies übernimmt  \texit{PhotosController#showPhotos}. 


Der Controller deklariert eine Abhängigkeit auf den \textit{PhotosClient}, über den sich der HTTP GET Aufruf an den Server durchführen lässt. Die Antwort ist ein Array vom Photo Metadaten mit der URL zu dem eigentlichen File, Titel, Beschreibung und diversen Kameraeinstellungen in JSON Format. JSON wird in JavaScript durch einen Aufruf von \texit{JSON.parse} direkt in ein Objekt umgewandelt. Der Angular HTTP Client macht es automatisch. Da die Darstellungslogik in den Controllern (Presentern) liegt, werden diese Rohdaten Objekte direkt als Models verwendet ohne extra Klassen dafür anzulegen.





