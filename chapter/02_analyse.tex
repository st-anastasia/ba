\chapter{Analyse}

\section{Einleitung}

In dem Kapitel \hyperref[sec:motivation]{Motivation} wurde geschildert, dass eine Web Photoverwaltung Software als Exempel für die Zielsetzungen dieser Arbeit dienen wird. Nun soll analysiert werden welche Realanforderungen notwendig sind um den zuvor definierten \hyperref[sec:zielsetzung]{Zielsetzungen} gerecht zu werden.

\section{UX}

\subsection{Schnelles Feedback}

Die Hauptanforderung der Webanwendung besteht darin dem Benutzer das an eine
native Applikation angelehnte Nutzungserlebnis zu gewährleisten. Die bei einer
klassischen Webanwendung entstehenden Ladezeiten, welche nach jeder
Benutzerinteraktion durch das erneute Laden und Darstellen des gesamten Inhaltes
auftreten sollen vermieden werden.

\subsection{Flaches Design}

\subsection{Responsive Design}

\subsection{Photo Zentrierung}

\section{User Stories}

\subsection{Authentifizierung}

Die Photosoftware soll Benutzern nur anhand einer Benutzerkennung und Passworts 
den Zugang gewähren. Unberechtigter Zugriff soll logischer Weise mit entsprechender Fehlermeldung verweigert werden.

\subsection{Navigation Menü}

Dem Benutzer soll im Stande sein in der Anwendung zwischen den Hauptfunkionen der Anwendung aus jedem beliebigen Unterbereich zu navigieren.

\subsection{Photo Galerie}
\label{ssec:photo_galerie}

Dem Benutzer soll eine Auflistung seiner gespeicherten Photos dargestellt werden.

\subsection{Paginierung/Nachladen der Photos}

Falls sich sehr viele Photos in der \hyperref[ssec:photo_galerie]{Photo Galerie} befinden, sollen diese nicht auf alle auf ein Mal geladen werden, damit die Anwendung nicht überladen wird. Stattdessen soll zuerst eine bestimmte Anzahl der Photos dargestellt werden und es anschließend dem Benutzer ermöglicht werden weitere Bilder in Batches nachzuladen.

\subsection{Photo Details}

Der Benutzer soll in der Lage sein, Photos mit Metadaten wie Name und Beschreibung zu annotieren. Bei der Auswahl in der eines einzelnen Photos in 
der Galerie sollen diese annotierten Photo Information dargestellt werden.

\subsection{Photo Großansicht}

Dem Benutzer soll ermöglicht werden ein bestimmtes Photo in der vollständiger Größe zu betrachten.

\subsection{Photo Freitext Suche}
\label{ssec:photo_freitext_suche}

Der Benutzer soll in Beschreibungen und Namen nach seinen Photos durch Eingabe von Freitext suchen können. Das Resultat der Suche soll ebenfalls wie die Photogalerie paginiert werden.

\subsection{Autovervollständigung der Freitextsuche}

Die Freitext Suche aus \ref{ssec:photo_freitext_suche} soll aus den in der gesamten Photosammlung befindenden Benamungen und Beschreibungen während der 
Sucheingabe vervollständigt werden.

\subsection{Photos nach Erstellungsdatum Filtern}

Der Benutzer soll in der Lage sein nur Photos aus einem von ihm gewählten Zeitraum zu betrachten. 



