\chapter{Analyse}
\label{sec:analysis}

\section{Einleitung}

In dem Kapitel \hyperref[sec:motivation]{Motivation} wurde geschildert, dass eine Web-Fotoverwaltungssoftware als Exempel für die Zielsetzungen dieser Arbeit dienen wird. Nun soll analysiert werden, welche Realanforderungen notwendig sind, um den zuvor definierten \hyperref[sec:zielsetzung]{Zielsetzungen} gerecht zu werden.

\section{User Experience}

Folgend werden Anforderungen beschrieben, die sich mit dem Nutzungserlebnis und der visuellen Gestaltung der Applikation befassen.

\subsection{Schnelles Feedback}

Die Hauptanforderung an die Webanwendung besteht darin, dem Benutzer das an eine native Applikation angelehnte Nutzungserlebnis zu gewährleisten. Die bei einer
klassischen Webanwendung entstehenden Ladezeiten, welche nach jeder Benutzerinteraktion durch das erneute Laden und Darstellen des gesamten Inhaltes
auftreten, sollen vermieden werden.

\subsection{Flaches Design}

Das Benutzerinterface der Anwendung soll unter Anwendung der Paradigmen vom flachen Design minimalistisch, jedoch mit klar erkennbaren Aktionsaufrufen, gestaltet werden.

\subsection{Responsive Design}

Die Webanwendung soll sich auf potentiell unterschiedliche Displaygrößen anpassen. Das Benutzerinterface soll dabei nicht komplett für jede mögliche Abstufung der Displaygröße neu gestaltet werden. Für das Layout sollen Regeln verwendet werden, die es dem gleichen Interface erlauben, seine Elemente bei schrumpfender bzw. wachsender Größe neu zu positionieren.

\subsection{Fotozentrierung}
\label{sec:spec:photo_centering}

Ein besonderes Unterproblem der Adaptierung an verschiedene Gerätedisplays ist die Fotobetrachtung. Hier sind sowohl die Auflösung des Fotos als auch des Displays für die Software nicht zu Implementierungszeit, sondern erst zur Laufzeit bekannt. Um das Foto in der Gesamtheit auf einem beliebigen Display zu betrachten, soll es zur Laufzeit sowohl in der Detailansicht als auch innerhalb seines Platzhalters in der Gesamtübersicht zentriert werden.

\section{User Stories}

In diesem Unterkapitel werden die einzelnen Features der Beispiel-Applikation definiert.

\subsection{Authentifizierung}
\label{sec:spec:authentication}

Die Fotosoftware soll den Benutzern nur anhand einer Benutzerkennung und eines Passworts den Zugang gewähren.

\subsection{Navigation Menü}
\label{sec:spec:menu}

Der Benutzer soll im Stande sein zwischen den Hauptfunkionen der Anwendung aus jedem beliebigen Unterbereich zu navigieren.

\subsection{Foto Galerie}
\label{sec:spec:photo_gallery}

Dem Benutzer soll eine Auflistung seiner gespeicherten Fotos dargestellt werden.

\subsection{Paginierung/Nachladen der Fotos}
\label{sec:spec:pagination}

Falls sich sehr viele Fotos in der \hyperref[sec:spec:photo_gallery]{Foto Galerie} befinden, sollen diese nicht im selben Augenblick geladen werden, damit die Anwendung nicht überlastet wird. Stattdessen soll zuerst eine bestimmte Anzahl der Fotos dargestellt werden und es anschließend dem Benutzer ermöglicht werden, weitere Bilder stapelweise nachzuladen.

\subsection{Foto Freitext Suche}
\label{sec:spec:photo_search}

Der Benutzer soll in Beschreibungen und Namen nach seinen Fotos durch Eingabe von Freitext suchen können. Das Resultat der Suche soll ebenfalls wie die Fotogalerie paginiert werden.

\subsection{Fotos nach Erstellungsmonat gruppieren}
\label{sec:spec:photo_groups}

Dem Benutzer soll ermöglicht werden, die Fotos nach Erstellungsmonat gruppiert zu betrachten. Die Darstellung der Monatsgruppen erfolgt in absteigender Reihenfolge.

\subsection{Foto Details}
\label{sec:spec:photo_details}

Der Benutzer soll in der Lage sein, Fotos mit Metadaten wie Name und Beschreibung zu annotieren. Bei der Auswahl eines einzelnen Fotos in der Galerie sollen diese annotierten Fotoinformationen veranschaulicht werden.

\subsection{Foto Großansicht}

Dem Benutzer soll ermöglicht werden, ein bestimmtes Foto in der vollständiger Größe zu betrachten.

\subsection{Foto Slider}
\label{sec:spec:photo_slider}

Wenn der Benutzer die Detailansicht eines Fotos aus der Galerie auswählt, soll es ihm ferner möglich sein aus der Detailansicht zum nächsten oder dem vorherigen Foto zu navigieren.
